\newif\ifshowsolutions
\showsolutionstrue
\documentclass{article}
\usepackage{listings}
\usepackage{amsmath}
%\usepackage{subfigure}
\usepackage{subfig}
\usepackage{amsthm}
\usepackage{amsmath}
\usepackage{amssymb}
\usepackage{graphicx}
\usepackage{mdwlist}
\usepackage[colorlinks=true]{hyperref}
\usepackage{geometry}
\usepackage{titlesec}
\geometry{margin=1in}
\geometry{headheight=2in}
\geometry{top=2in}
\usepackage{palatino}
\usepackage{mathrsfs}
\usepackage{fancyhdr}
\usepackage{paralist}
\usepackage{todonotes}
\setlength{\marginparwidth}{2.15cm}
\usepackage{tikz}
\usetikzlibrary{positioning,shapes,backgrounds}
\usepackage{float} % Place figures where you ACTUALLY want it
\usepackage{comment} % a hack to toggle sections
\usepackage{ifthen}
\usepackage{mdframed}
\usepackage{verbatim}
\usepackage[strings]{underscore}
\usepackage{listings}
\usepackage{bbm}
\rhead{}
\lhead{}

\renewcommand{\baselinestretch}{1.15}

% Shortcuts for commonly used operators
\newcommand{\E}{\mathbb{E}}
\newcommand{\Var}{\operatorname{Var}}
\newcommand{\Cov}{\operatorname{Cov}}
\newcommand{\Bias}{\operatorname{Bias}}
\DeclareMathOperator{\argmin}{arg\,min}
\DeclareMathOperator{\argmax}{arg\,max}

% do not number subsection and below
\setcounter{secnumdepth}{1}

% custom format subsection
\titleformat*{\subsection}{\large\bfseries}

% set up the \question shortcut
\newcounter{question}[section]
\newenvironment{question}[1][]
  {\refstepcounter{question}\par\addvspace{1em}\textbf{Question~\Alph{question}\!
    \ifthenelse{\equal{#1}{}}{}{ [#1 points]}: }}
    {\par\vspace{\baselineskip}}

\newcounter{subquestion}[question]
\newenvironment{subquestion}[1][]
  {\refstepcounter{subquestion}\par\medskip\textbf{\roman{subquestion}.\!
    \ifthenelse{\equal{#1}{}}{}{ [#1 points]:}} }
  {\par\addvspace{\baselineskip}}

\titlespacing\section{0pt}{12pt plus 2pt minus 2pt}{0pt plus 2pt minus 2pt}
\titlespacing\subsection{0pt}{12pt plus 4pt minus 2pt}{0pt plus 2pt minus 2pt}
\titlespacing\subsubsection{0pt}{12pt plus 4pt minus 2pt}{0pt plus 2pt minus 2pt}


\newenvironment{hint}[1][]
  {\begin{em}\textbf{Hint: }}{\end{em}}

\ifshowsolutions
  \newenvironment{solution}[1][]
    {\par\medskip \begin{mdframed}\textbf{Solution~\Alph{question}#1:} \begin{em}}
    {\end{em}\medskip\end{mdframed}\medskip}
  \newenvironment{subsolution}[1][]
    {\par\medskip \begin{mdframed}\textbf{Solution~\Alph{question}#1.\roman{subquestion}:} \begin{em}}
    {\end{em}\medskip\end{mdframed}\medskip}
\else
  \excludecomment{solution}
  \excludecomment{subsolution}
\fi

\newcommand{\boldline}[1]{\underline{\textbf{#1}}}

\chead{%
  {\vbox{%
      \vspace{2mm}
      \large
      Machine Learning \& Data Mining \hfill
      Caltech CS/CNS/EE 155 \hfill \\[1pt]
      Miniproject 1\hfill
      Released January $28^{th}$, 2017 \\
    }
  }
}

\begin{document}
\pagestyle{fancy}

% LaTeX is simple if you have a good template to work with! To use this document, simply fill in your text where we have indicated. To write mathematical notation in a fancy style, just write the notation inside enclosing $dollar signs$.

% For example:
% $y = x^2 + 2x + 1$

% For help with LaTeX, please feel free to see a TA!



\section{Introduction}
\medskip
\begin{itemize}

    \item \boldline{Group members} \\
    Vaibhav Anand \\
    Nikhil Gupta \\
    Michael Hashe \\
    
    \item \boldline{Team name} \\
    The Breakfast Club

    \item \boldline{GitHub link} \\
    \href{https://github.com/nkgupta1/CS155-Project1-Voter}{https://github.com/nkgupta1/CS155-Project1-Voter}
    
    \item \boldline{Division of labour} \\
    % Insert text here.

\end{itemize}



\section{Overview}
\medskip
\begin{itemize}

    \item \boldline{Models and techniques tried}
    \begin{itemize}
    % Insert text here. Bullet points can be made using '\item'. Models and techniques should be bolded using '\textbf{}'.
    \item \textbf{Neural Network} 

    \item \textbf{Random Forest} We experiemented with various types of random forest-based classifiers. In addition to the standard random forest model contained in sklearn, we examined boosted and bagged trees, as well as a feed-forward stack of random forest classifiers.

    \item \textbf{Linear Models}

    \item \textbf{Mixed Ensembles}
    \end{itemize}

    \item \boldline{Work timeline}
    \begin{itemize}
    % Insert text here. Bullet points can be made using '\item'.
    \item \textbf{Bullet:} Bullet text.
    \end{itemize}

\end{itemize}



\section{Approach}
\medskip
\begin{itemize}

    \item \boldline{Data processing and manipulation}
    \begin{itemize}
    % Insert text here. Bullet points can be made using '\item'.
    \item \textbf{Original Data:} At first, we trained the models on the raw data. These models ended up performing relatively poorly on the leaderboard. In looking closer at the data, we realized this was because of the way the data was labeled: questions with categorical answers had integers as labels for the different categories. As such, the models were trying to put an ordering to the dimensions where none existed, which hurt the accuracy of the models.
    \item \textbf{All Categories:} The first thing we tried in order to resolve this issue was to flatten the whole data set and make it all categorical. In particular, this choice was motived by a cursory look through the handbook, from which we noticed that a majority of the dimensions were categorical. Before we invested time in splitting the data into ordered and unordered dimensions, we tried training models when the whole data was made into categories. Models trained on this data had memory issues because of the size of the data set and there a very little chance of generalization because of the number of categories present.
    \item \textbf{Categories and Ordered:} After taking a closer look through the handbook, we realized that many of the issues we were having with categorizing the data was due to the specific characteristics of the dimensions. While most of the dimensions were unordered, many of them still had an ordering. As such, this forced us to manually sift through the data in order to bin the dimensions into ordered and unordered. In order to ease this process, we wrote a parser so that we could later modify which dimension were kept and how they were treated. 

    The range of each of the ordered dimensions was mapped to [0,1] so that the models could better deal with them. This choice was made after trying to normalize the data and having some of the values still being very large. The unordered data was made into categories with one category for each of the values present. The parser kept track of the way each dimension was processed so that the test data could be processed in the same way. This raised an issue of how to treat values for dimensions that were in the test set but not in the training set for the unordered dimensions (while this occurred, it was not particularly frequent). We ended up putting no value for this because we thought it would be better to omit data than to lie about the data. 

    After putting the data through this processing, the number of resulting dimension was around 50000, which is too many for the number of points that we had. In looking through how many dimensions the processor made each of the original dimensions into, there was one dimension that accounted for all but about 4000 of the dimensions, a unique ID for each family. As such, we added another option to the parser to completely delete dimensions. Models trained on this new, processed data set performed much better than the models performed on the original data set. We used NumPy for much of the data processing because of the number of methods present in the NumPy library which meant that we did not have to reimplement many of the functions. Furthermore, NumPy methods run very quickly on the NumPy arrays. 
    \item \textbf{Trimming Dimensions:} After using the last data set for most of the models, we were not able to improve on accuracy much more so we thought this might be due to noise in the dimensions of the data set. As such, we made the decision to go through the handbook more carefully and make the decision to keep or delete each of the dimensions. This is because there were a significant number of dimensions that added noise to the dataset. For example, there were many dimensions for line number which was not something that the people actually answered, so were something that we decided were insignificant. After this final trimming process, the number of dimensions present were reduced from about 4000 to 3000. This reduced dimensionality improved the performance of some of the models.
    \end{itemize}

    \item \boldline{Details of models and techniques}
    \begin{itemize}
    % Insert text here. Bullet points can be made using '\item'.
    \item \textbf{Bullet:} Bullet text.

    % If you would like to insert a figure, you can just use the following five lines, replacing the image path with your own and the caption with a 1-2 sentence description of what the image is and how it is relevant or useful.
    % \begin{figure}[H]
    % \centering
    % \includegraphics[width=\textwidth]{smiley.png}
    % \caption{Insert caption here.}
    % \end{figure}

    \end{itemize}

\end{itemize}



\section{Model Selection}
\medskip
\begin{itemize}

    \item \boldline{Scoring} \\
    % Insert text here.

    \item \boldline{Validation and Test} \\
    % Insert text here.

\end{itemize}



\section{Conclusion}
\medskip
\begin{itemize}

    \item \boldline{Discoveries} \\
    The main discovery from this competition was the difficulty of fitting a ``hard" and imbalanaced dataset, and in particular the challenges accompanying creating a model both sufficiently able to fit a training dataset (2008 data) and sufficiently general to adapt to an unknown and distinct new dataset (2012 data). These challenges will be analyzed in more depth in the section below.\\

    In terms of models used, we experimented with variants of common models examined in class. In particular, we utilized variations of trees aside from the default CART tree and various linear models. While our best performing models remained standard neural networks (and, on the test dataset, random forests), we certainly learned more about the variants of these models currently in use.\\

    In examining the occasionally massive changes in the leaderboard between the first and second parts of the competition, and in examining the differing performances of our own models on these two datasets, we also discovered the dangers of overfitting and the propensities of various models for doing so. In particular, we found (after submission) that our more ``general" models (i.e., collections of networks and forests) tended to perform reasonably well on the new dataset, whereas models more prone to overfitting (smaller forests and networks, linear models) tended to suffer more. Considering the large drops experienced by many teams initially above us on the leaderboard, it is our belief that this ``overfitting the leaderboard" was in fact a fairly general problem.

    \item \boldline{Challenges} \\
    As mentioned above, the main problem experienced in the competition was the dataset. In particular, we found that the dataset was lopsided (~75\% of those polled voted), rather different from the test set (in which ~65\% of those polled voted), and very noisy (i.e., even the best/most overfitted models were unable to break 80\% classification accuracy). In dealing with such a dataset, we found that most models performed within a percent or two of each other, which made distinguishing the best models somewhat difficult. Further, it meant that it was very easy to fall into a trap of overfitting (i.e., gridsearching parameters, submitting an excessive number of models, and falsely concluding that the models that best fit the data did so because they were the best models). We managed to avoid this reasonably well, although we could have improved on our part 2 score had we selected better. Were we to do this competition again, we would have put more thought into how to best select models.\\

    Within the dataset, we also found that many of the parameters were either repeated, arbitrary (i.e., identification numbers), or answered for only a very small number of responders (i.e., several questions focused very specifically on certain subsets of the population, and therefore had response rates in the hundreds). We curated the dataset and removed many of these parameters from consideration, although determining whether or not a parameter should be included was not always a clear decision. Were we to redo this competition, we would have curated the dataset in a more principled (and, ideally, automated) manner, through some manner of cross-validation on model performance with various parameters removed. We would also likely have removed even more parameters; it is our belief that this might help with the generalization accuracy of the model.\\

    TENSORFLOW\\

    We additionally encountered problems with resource limitations in training large models. While most of our models trained in reasonable time, we found that some (in particular, certain variants of random forests and SVMs) were slow enough to be infeasible.\\

    \item \boldline{Concluding Remarks} \\
    % Insert text here.

\end{itemize}



\end{document}